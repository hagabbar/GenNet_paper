% ****** Start of file apssamp.tex ******
%
%   This file is part of the APS files in the REVTeX 4.1 distribution.
%   Version 4.1r of REVTeX, August 2010
%
%   Copyright (c) 2009, 2010 The American Physical Society.
%
%   See the REVTeX 4 README file for restrictions and more information.
%
% TeX'ing this file requires that you have AMS-LaTeX 2.0 installed
% as well as the rest of the prerequisites for REVTeX 4.1
%
% See the REVTeX 4 README file
% It also requires running BibTeX. The commands are as follows:
%
%  1)  latex apssamp.tex
%  2)  bibtex apssamp
%  3)  latex apssamp.tex
%  4)  latex apssamp.tex
%
\documentclass[%
%superscriptaddress,
%groupedaddress,
%unsortedaddress,
%runinaddress,
%frontmatterverbose, 
%preprint,
showpacs,
%preprintnumbers,
%nofootinbib,
%nobibnotes,
%bibnotes,
 amsmath,amssymb,
 aps,
 twocolumn,
 prl,
 reprint,
%pra,
%prb,
%rmp,
%prstab,
%prstper,
floatfix,
]{revtex4-1}

\usepackage{graphicx}% Include figure files
\usepackage{dcolumn}% Align table columns on decimal point
\usepackage{bm}% bold math
%\usepackage{lineno}
\usepackage{color}
\usepackage{acronym}
\usepackage{multirow}
\usepackage{tabularx}
\usepackage{hyperref}
%\addbibresource{references.bib}
%\linenumbers % Commence numbering lines
%\usepackage{hyperref}% add hypertext capabilities
%\usepackage[mathlines]{lineno}% Enable numbering of text and display math
%\linenumbers\relax % Commence numbering lines
\hypersetup{
%--- fill inside borders ---
  colorlinks=true,        % false: boxed links; true: colored links
  linkcolor=black,         % color of internal links
  citecolor=cyan,         % color of links to bibliography
}

%\usepackage[showframe,%Uncomment any one of the following lines to test 
%%scale=0.7, marginratio={1:1, 2:3}, ignoreall,% default settings
%%text={7in,10in},centering,
%%margin=1.5in,
%%total={6.5in,8.75in}, top=1.2in, left=0.9in, includefoot,
%%height=10in,a5paper,hmargin={3cm,0.8in},
%]{geometry}

%% ----- some handy shortcuts
\newcommand{\dcc}{LIGO-P1700378}
\newcommand{\optsnr}{\rho_{\mathrm{opt}}}
\newcommand{\fmin}{f_{\mathrm{min}}}

%% ----- comment commands for each of us
\newcommand{\chris}[1]{\textbf{\textcolor{green}{CHRIS: #1}}}
\newcommand{\michael}[1]{\textbf{\textcolor{red}{MICHAEL: #1}}}
\newcommand{\hunter}[1]{\textbf{\textcolor{blue}{HUNTER: #1}}}
\newcommand{\fergus}[1]{\textbf{\textcolor{cyan}{FERGUS: #1}}}

%% ----- result macros - where we store all key results
\newcommand{\cnnsnreight}{97.88}

%% ----- input git-version tag
%\input{tag.tex}

\begin{document}

\preprint{APS/123-QED}

%
% Be clear and specific. Do not claim too much or too little.
%
\title{Any title, doesnt really matter}

\author{Hunter Gabbard}
 \email{Corresponding author: h.gabbard.1@research.gla.ac.uk}
\author{Chris Messenger}
\affiliation{
 SUPA, School of Physics and Astronomy, \\
 University of Glasgow, \\
 Glasgow G12 8QQ, United Kingdom \\
}

\date{\today}% It is always \today, today,
             %  but any date may be explicitly specified

%\date{\commitDATE\\\mbox{\small \commitID}\\\mbox{\dcc}}

%
% Explain what the result is and why it’s important, plus possibly a sentence
% or two of introduction, motivation, methods, caveats.
%
\begin{abstract} 
%
This may be an abstract one day.
%
\end{abstract}

%\pacs{04.30.-w, 04.80.Nn}% PACS, the Physics and Astronomy


                             % Classification Scheme.
%\keywords{Suggested keywords}%Use showkeys class option if keyword
                              %display desired


\maketitle

\acrodef{BBH}[BBH]{binary black hole}
\acrodef{SNR}[SNR]{signal-to-noise ratio}
\acrodef{PSD}[PSD]{power spectral density}
\acrodef{FFT}[FFT]{fast Fourier transform}
\acrodef{CNN}[CNN]{convolutional neural network}
\acrodef{ROC}[ROC]{receiver operator characteristic}

%\tableofcontents

%
% Explain what the result is and why it’s important, particularly arguing how
% the paper will move physics forward. Like the abstract, but shorter and with
% a focus on WHY not HOW.
%

%
% Give sufficient background so the general reader can understand what you did
% and why you did it.
%
\textit{Introduction.}--- 

%
% intro to gravitational-waves
With the recent detections of [some number] binary black holes \cite{detection papers} and binary neutron merger GW170817 \cite{BNS paper} by the LIGO-Virgo Scientific Collaboration (LVC), the era of gravitational-wave astronomy has begun. The LVC is composed of three detectors in Hanford, Washington State, Livingston, Louisiana, and Pisa, Italy. Once the detectors will have reached design senstivity, they will be able probe a search distance on the order of 200 Mpc. As the detectors become more sensitive, we will be limited by the sheer number of signals to process. As such, it is of paramount importance that we are able to get full parameter estimate posteriors on signals in a timely and efficient manner.

%
% intro to Bayesian Parameter Estimation
%
In order to determine the parameters of a detection, we apply Bayes' theorem \cite{Bayes and R. Price, Phil. Trans. R. Soc. London 53, 370 (1763). E. T. Jaynes, Probability Theory: The Logic of Science, edited by G. L. Bretthorst (Cambridge University Press, Cambridge, England, 2003).}. We can gain most information concerning the parameters of a source through the probability density function (PDF) of all unknown parameters, given data from both detectors. 

The PDF posterior is computed from a liklihood of the data given given some parameters multiplied by a prior PDF on  the parameters made independent of the recorded data. We can compute marginalized PDFs using a suite of algorithms from C/C++ package known as \texttt{LALInference} \cite{LALSuite}. Two independent stochastic sampling techniques are utilized in \texttt{LALInference} known as Markov-chain Monte Carlo \cite{MCMC paper} and nested sampling \cite{nested sampling}.

   

%
% intro to generative adversarial networks
%
A variant of machine learning, generative adversarial networks (GANs), has gained some traction in recent years \cite{arxiv:1406.2661}. Some successful applications of GANs can be seen in image retrieval of historical archives \cite{arxiv:1607.02748}, text to image synthesis \cite{arxiv:1605.05396}, simulation of high energy particle showers \cite{arxiv:1712.10321}, along with many others. 

GANs are composed of two neural networks, a discriminator and a generator. The generator network attempts to map random latent variables to some distribution which the user wants the GAN to approximate, whereas the discriminator network will attempt to distinguish between samples from the real distribution and fake samples made by the generator network. You need only to feed into the generator as input a vector of latent random variables (can be uniform, Gaussian, etc.). The variables are latent in that they are not directly related to the distribution that the generator is trying to emulate, however in our case they do have an underlying Gaussian distribution. An additional motivation for using GANs is that they are also semi-supervised, so few training samples are needed, and you are left with by default two tools (one for classification and one for sample generation).  

We use a variant of a GAN called deep convolutional generative adversarial
networks (DCGAN)~\cite{1511.06434}. DCGANs are composed of two networks which act as
the generator and discriminator, however the difference between DCGANs and their more standard
GAN brethren is that both networks are entirely made of
convolutional layers. We follow the architecture guidelines laid out in Radford et al.
~\cite{1511.06434} which are as follows: replace pooling (downsampling) layers with strided
convolutions (alternative to pooling),
use batch normalization in both the generator and the discriminator, remove all fully
connected hidden layers, use ReLU activation functions~\cite{Nair:2010:RLU:3104322.3104425}
in the generator except for the output,
and use LeakyReLU activation functions~\cite{Maas2013RectifierNI} in the discriminator for all layers.

To-do:

\begin{itemize}
\item GWs.
\item Bayesian Parameter Estimation.
\end{itemize}

\textit{Methods}---


To-do:

\begin{itemize}
\item How do you incorporate priors.
\item Convergence.
\item Volume of training data.
\item Modifications from standard GAN (most of section).
\end{itemize}

\textit{Results}---


To-do:

GW150914. Show PE estimates on mass, spins, etc.

Small section on waveform reconstruction. Plot of this as well.  

\begin{itemize}
\item Which priors were used. Same as GW150914 analysis.
\end{itemize}
\textit{Conclusions}---

To-do:

Percision that we get, speed (sell this hard), 

\begin{itemize}
\item Sell speed, with the right caveats.
\item Waveform reconstruction.
\item We are not model indepenent.
\item Non-Gaussian noise.
\end{itemize}

%
% acknowledge peopkle and funding agencies
%
\emph{Acknowledgements.}---
%
We would like to acknowledge valuable input from the LIGO-Virgo Collaboration
specifically from T.~Dent, R.~Reinhard, I.~Siong Heng, M.~Cavalgia, and the compact binary coalescence and
machine-learning working groups. The authors also gratefully acknowledge the
Science and Technology Facilities Council of the United Kingdom. CM is
supported by the Science and Technology Research Council (grant
No.~ST/~L000946/1).
%
%\end{acknowledgments}


% The \nocite command causes all entries in a bibliography to be printed out
% whether or not they are actually referenced in the text. This is appropriate
% for the sample file to show the different styles of references, but authors
% most likely will not want to use it.

\bibliographystyle{apsrev4-1}
\bibliography{references}% Produces the bibliography via BibTeX.


\end{document}
%
% ****** End of file apssamp.tex ******
