% ****** Start of file apssamp.tex ******
%
%   This file is part of the APS files in the REVTeX 4.1 distribution.
%   Version 4.1r of REVTeX, August 2010
%
%   Copyright (c) 2009, 2010 The American Physical Society.
%
%   See the REVTeX 4 README file for restrictions and more information.
%
% TeX'ing this file requires that you have AMS-LaTeX 2.0 installed
% as well as the rest of the prerequisites for REVTeX 4.1
%
% See the REVTeX 4 README file
% It also requires running BibTeX. The commands are as follows:
%
%  1)  latex apssamp.tex
%  2)  bibtex apssamp
%  3)  latex apssamp.tex
%  4)  latex apssamp.tex
%
\documentclass[%
%superscriptaddress,
%groupedaddress,
%unsortedaddress,
%runinaddress,
%frontmatterverbose, 
%preprint,
showpacs,
%preprintnumbers,
%nofootinbib,
%nobibnotes,
%bibnotes,
 amsmath,amssymb,
 aps,
 twocolumn,
 prl,
 reprint,
%pra,
%prb,
%rmp,
%prstab,
%prstper,
floatfix,
]{revtex4-1}

\usepackage{graphicx}% Include figure files
\usepackage{dcolumn}% Align table columns on decimal point
\usepackage{bm}% bold math
%\usepackage{lineno}
\usepackage{color}
\usepackage{acronym}
\usepackage{multirow}
\usepackage{tabularx}
\usepackage{hyperref}
%\addbibresource{references.bib}
%\linenumbers % Commence numbering lines
%\usepackage{hyperref}% add hypertext capabilities
%\usepackage[mathlines]{lineno}% Enable numbering of text and display math
%\linenumbers\relax % Commence numbering lines
\hypersetup{
%--- fill inside borders ---
  colorlinks=true,        % false: boxed links; true: colored links
  linkcolor=black,         % color of internal links
  citecolor=cyan,         % color of links to bibliography
}

%\usepackage[showframe,%Uncomment any one of the following lines to test 
%%scale=0.7, marginratio={1:1, 2:3}, ignoreall,% default settings
%%text={7in,10in},centering,
%%margin=1.5in,
%%total={6.5in,8.75in}, top=1.2in, left=0.9in, includefoot,
%%height=10in,a5paper,hmargin={3cm,0.8in},
%]{geometry}

%% ----- some handy shortcuts
\newcommand{\dcc}{LIGO-P1700378}
\newcommand{\optsnr}{\rho_{\mathrm{opt}}}
\newcommand{\fmin}{f_{\mathrm{min}}}

%% ----- comment commands for each of us
\newcommand{\chris}[1]{\textbf{\textcolor{green}{CHRIS: #1}}}
\newcommand{\michael}[1]{\textbf{\textcolor{red}{MICHAEL: #1}}}
\newcommand{\hunter}[1]{\textbf{\textcolor{blue}{HUNTER: #1}}}
\newcommand{\fergus}[1]{\textbf{\textcolor{cyan}{FERGUS: #1}}}

%% ----- result macros - where we store all key results
\newcommand{\cnnsnreight}{97.88}

%% ----- input git-version tag
%\input{tag.tex}

\begin{document}

\preprint{APS/123-QED}

%
% Be clear and specific. Do not claim too much or too little.
%
\title{Any title, doesnt really matter}

\author{Hunter Gabbard}
 \email{Corresponding author: h.gabbard.1@research.gla.ac.uk}
\author{Chris Messenger}
\affiliation{
 SUPA, School of Physics and Astronomy, \\
 University of Glasgow, \\
 Glasgow G12 8QQ, United Kingdom \\
}

\date{\today}% It is always \today, today,
             %  but any date may be explicitly specified

%\date{\commitDATE\\\mbox{\small \commitID}\\\mbox{\dcc}}

%
% Explain what the result is and why it’s important, plus possibly a sentence
% or two of introduction, motivation, methods, caveats.
%
\begin{abstract} 
%
This may be an abstract one day.
%
\end{abstract}

%\pacs{04.30.-w, 04.80.Nn}% PACS, the Physics and Astronomy


                             % Classification Scheme.
%\keywords{Suggested keywords}%Use showkeys class option if keyword
                              %display desired


\maketitle

\acrodef{BBH}[BBH]{binary black hole}
\acrodef{SNR}[SNR]{signal-to-noise ratio}
\acrodef{PSD}[PSD]{power spectral density}
\acrodef{FFT}[FFT]{fast Fourier transform}
\acrodef{CNN}[CNN]{convolutional neural network}
\acrodef{ROC}[ROC]{receiver operator characteristic}

%\tableofcontents

%
% Explain what the result is and why it’s important, particularly arguing how
% the paper will move physics forward. Like the abstract, but shorter and with
% a focus on WHY not HOW.
%

%
% Give sufficient background so the general reader can understand what you did
% and why you did it.
%
\textit{Introduction.}--- 
%
% intro to gravitational-waves
\begin{itemize}
\item Bayesian Parameter Estimation.
\item Generative Adversarial Networks.
\item GWs are the shit.
\end{itemize}

\textit{Methods}---

\begin{itemize}
\item How do you incorporate priors.
\item Convergence.
\item Volume of training data.
\item Modifications from standard GAN (most of section).
\end{itemize}

\textit{Results}---

GW150914. Show PE estimates on mass, spins, etc.

Small section on waveform reconstruction. Plot of this as well.  

\begin{itemize}
\item Which priors were used. Same as GW150914 analysis.
\end{itemize}
\textit{Conclusions}---

Percision that we get, speed (sell this hard), 

\begin{itemize}
\item Sell speed, with the right caveats.
\item Waveform reconstruction.
\item We are not model indepenent.
\item Non-Gaussian noise.
\end{itemize}

%
% acknowledge peopkle and funding agencies
%
\emph{Acknowledgements.}---
%
We would like to acknowledge valuable input from the LIGO-Virgo Collaboration
specifically from T.~Dent, R.~Reinhard, I.~Siong Heng, M.~Cavalgia, and the compact binary coalescence and
machine-learning working groups. The authors also gratefully acknowledge the
Science and Technology Facilities Council of the United Kingdom. CM is
supported by the Science and Technology Research Council (grant
No.~ST/~L000946/1).
%
%\end{acknowledgments}


% The \nocite command causes all entries in a bibliography to be printed out
% whether or not they are actually referenced in the text. This is appropriate
% for the sample file to show the different styles of references, but authors
% most likely will not want to use it.

\bibliographystyle{apsrev4-1}
\bibliography{references}% Produces the bibliography via BibTeX.


\end{document}
%
% ****** End of file apssamp.tex ******
